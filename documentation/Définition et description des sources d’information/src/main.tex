\documentclass[a4paper, 12pt]{article}

\usepackage{bera}% optional: just to have a nice mono-spaced font
\usepackage{listings}
\usepackage{xcolor}
\usepackage{tikz}

\usepackage{listings}
\usepackage{color}
\usepackage{booktabs, makecell, colortbl}

\definecolor{dkgreen}{rgb}{0,0.6,0}
\definecolor{gray}{rgb}{0.5,0.5,0.5}
\definecolor{mauve}{rgb}{0.58,0,0.82}
\definecolor{gray}{rgb}{0.4,0.4,0.4}
\definecolor{darkblue}{rgb}{0.0,0.0,0.6}
\definecolor{lightblue}{rgb}{0.0,0.0,0.9}
\definecolor{cyan}{rgb}{0.0,0.6,0.6}
\definecolor{darkred}{rgb}{0.6,0.0,0.0}


\colorlet{punct}{red!60!black}
\definecolor{background}{HTML}{EEEEEE}
\definecolor{delim}{RGB}{20,105,176}
\colorlet{numb}{magenta!60!black}

\lstdefinelanguage{json}{
    basicstyle=\normalfont\ttfamily,
    numbers=left,
    numberstyle=\scriptsize,
    stepnumber=1,
    numbersep=8pt,
    showstringspaces=false,
    breaklines=true,
    frame=lines,
    backgroundcolor=\color{background},
    literate=
     *{0}{{{\color{numb}0}}}{1}
      {1}{{{\color{numb}1}}}{1}
      {2}{{{\color{numb}2}}}{1}
      {3}{{{\color{numb}3}}}{1}
      {4}{{{\color{numb}4}}}{1}
      {5}{{{\color{numb}5}}}{1}
      {6}{{{\color{numb}6}}}{1}
      {7}{{{\color{numb}7}}}{1}
      {8}{{{\color{numb}8}}}{1}
      {9}{{{\color{numb}9}}}{1}
      {:}{{{\color{punct}{:}}}}{1}
      {,}{{{\color{punct}{,}}}}{1}
      {\{}{{{\color{delim}{\{}}}}{1}
      {\}}{{{\color{delim}{\}}}}}{1}
      {[}{{{\color{delim}{[}}}}{1}
      {]}{{{\color{delim}{]}}}}{1},
}

\lstset{
  basicstyle=\ttfamily\footnotesize,
  columns=fullflexible,
  showstringspaces=false,
  numbers=left,                   % where to put the line-numbers
  numberstyle=\tiny\color{gray},  % the style that is used for the line-numbers
  stepnumber=1,
  numbersep=5pt,                  % how far the line-numbers are from the code
  backgroundcolor=\color{white},      % choose the background color. You must add \usepackage{color}
  showspaces=false,               % show spaces adding particular underscores
  showstringspaces=false,         % underline spaces within strings
  showtabs=false,                 % show tabs within strings adding particular underscores
  frame=none,                   % adds a frame around the code
  rulecolor=\color{black},        % if not set, the frame-color may be changed on line-breaks within not-black text (e.g. commens (green here))
  tabsize=2,                      % sets default tabsize to 2 spaces
  captionpos=b,                   % sets the caption-position to bottom
  breaklines=true,                % sets automatic line breaking
  breakatwhitespace=false,        % sets if automatic breaks should only happen at whitespace
  title=\lstname,                   % show the filename of files included with \lstinputlisting;
                                  % also try caption instead of title  
  commentstyle=\color{gray}\upshape
}


\lstdefinelanguage{XML}
{
  morestring=[s][\color{mauve}]{"}{"},
  morestring=[s][\color{black}]{>}{<},
  morecomment=[s]{<?}{?>},
  morecomment=[s][\color{dkgreen}]{<!--}{-->},
  stringstyle=\color{black},
  identifierstyle=\color{lightblue},
  keywordstyle=\color{red},
  morekeywords={xmlns,xsi,noNamespaceSchemaLocation,type,id,x,y,source,target,version,tool,transRef,roleRef,objective,eventually}% list your attributes here
}

\usepackage[utf8]{inputenc}
\usepackage[T1]{fontenc}
\usepackage[french]{babel}
\usepackage{graphicx}
\usepackage{amsmath}
\usepackage{hyperref}
\usepackage{lmodern}
\usepackage{moreverb}
\usepackage{multicol}
% Please add the following required packages to your document preamble:
 \usepackage[table,xcdraw]{xcolor}
% If you use beamer only pass "xcolor=table" option, i.e. \documentclass[xcolor=table]{beamer}
 \usepackage[normalem]{ulem}
\useunder{\uline}{\ul}{}



\usepackage[a4paper,left=2cm,right=2cm,top=2cm,bottom=2cm]{geometry}

\pagestyle{headings}
\pagestyle{plain}


\setcounter{secnumdepth}{4}
\setcounter{tocdepth}{4}
\makeatletter


\makeatother



\makeatletter
\def\toclevel@subsubsubsection{4}
\def\toclevel@paragraph{5}
\def\toclevel@subparagraph{6}
\makeatother


\setlength{\parindent}{0cm}
\setlength{\parskip}{1ex plus 0.5ex minus 0.2ex}
\newcommand{\hsp}{\hspace{20pt}}
\newcommand{\HRule}{\rule{\linewidth}{0.5mm}}



\begin{document}

\begin{titlepage}
  \begin{sffamily}
  \begin{center}

   
  \textsc{\LARGE }\\[2cm]

    \textsc{\Large Définition et description des sources d’information}

    % Title
    \HRule \\[0.4cm]
    { \huge  \textsc{Projet d'interopérabilité} \\
    \textsc{\small Groupe 4}\\ [0.4cm] }
	

    \HRule \\[2cm]
    \textsc {Idriss BENGUEZZOU\\ Iness BOUABID\\Ali BOUGASSAA\\Ghilas MEZIANE }
 \begin{figure}
     \centering
    \includegraphics[scale=0.2]{logoUJM.png}
     \label{fig:ujm_logo}
 \end{figure}
   
    \

    \vfill

    % Bottom of the page
    {\large {} 08/02/2023}

  \end{center}
  \end{sffamily}
\end{titlepage}


\newpage
\tableofcontents

\newpage
\section{Objet du document}

L'objectif du document est de fournir une définition et une description détaillées des sources d'informations qui seront utilisées pour alimenter la base de données centralisée du projet de promotion de la vie culturelle dans la région.

Pour chaque source d'information, le document va identifier et décrire les types d'informations que la source fournit, le format dans lequel les informations sont disponibles, les moyens d'accéder à ces informations, ainsi que les méthodes et les outils utilisés pour extraire, nettoyer et stocker les données dans la base de données centrale.



\section{ Description des sources des données}
 \subsection{Liste et localisation des musées en France}

 Le site  \textbf{https://data.culture.gouv.fr/ }
 contient des informations sur la liste et la localisation des musées de France. Il contient plusieurs colonnes mais celles qui nous importe le plus sont les suivantes : 

 \begin{itemize}
     \item Le nom du musée.
     \item L'adresse.
     \item Le code postale.
     \item La ville.
     \item La latitude.
     \item La longitude.     
 \end{itemize}

Le jeu de données vient dans différents formats : des données au formats CSV, JSON, et des feuilles de calculs Excel.

  
 \subsection{Informations sur le patrimoine culturel en France}

 Le site \href{https://tourisme62.opendatasoft.com/}{\textbf{https://tourisme62.opendatasoft.com/}}
nous fourni des données sur le patrimoine culturel dans le département du Pas-de-Calais en France.  

Chaque instance de données comprend les informations suivantes :

\begin{itemize}
    \item Le nom du patrimoine culturel.
    \item Le type du patrimoine (site archéologique, monument historique...etc).
    \item La commune.
    \item Le code postal.
    \item L'adresse.
    \item La latitude.
    \item La longitude.
\end{itemize}

Le jeu de données vient dans 3 formats différents : JSON, CSV et Feuille de calcul Excel.
 
 \subsection{Informations sur la fréquentation des musées en France}
Chaque ligne du jeu de données du site \href{https://data.opendatasoft.com/}{\textbf{https://data.opendatasoft.com/}}
représente un musée et ses données de fréquentation pour une seule année, il contient aussi les informations sur les visiteurs.

Chaque ligne comporte les champs suivants : 
\begin{itemize}
    \item Le nom du musée.
    \item L'année.
    \item La fréquentation, soit le nombre de visiteur du musée selon l'année.
    \item L'adresse.
    \item La ville.
    \item Le code postal.
\end{itemize}

Les données sont représentées dans 3 formats : CSV, JSON et Excel.
 
 \subsection{Information sur le patrimoine mondial en France}

Le site \href{https://whc.unesco.org/}{\textbf{https://whc.unesco.org/}} contient les informations sur le patrimoine mondiale de l'UNESCO. Pour l'alimentation de la base de connaissances de Wikibase nous allons utiliser uniquement les données du patrimoine présent en France.

Le jeu de données est structures dans un seul format; XML et contient les informations détaillées sur chaque site du patrimoine. Chaque entrée, présente comme éléments :

\begin{itemize}
    \item  Le nom du site.
    \item Le pays ou se trouve le site.
    \item La date d'inscription sur la liste du patrimoine mondiale.
    \item Le statut de l'inscription.
    \item Les raisons pour lesquelles le site a été inclus dans la liste.
    \item La description du site.
    \item Les coordonnées géographiques.
\end{itemize}


\end{document}